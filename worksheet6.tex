\documentclass{exam}
\usepackage{amssymb}
\title{CS113/DISCRETE MATHEMATICS-SPRING 2024}
\author{Worksheet 6}
\date{Topic: Laws Of Inference For Quantified Statements}
\begin{document}
\maketitle

\begin{center}
\fbox{\fbox{\parbox{5.5in}{\centering Building upon your understanding of the laws of inference, we will now learn these  laws for quantified statements.Happy Learning!}}}
\end{center}

\vspace{5mm}
\makebox[0.75\textwidth]{Student's Name and ID:\enspace\hrulefill}

\vspace{5mm}
\makebox[0.75\textwidth]{Instructor’s name:\enspace\hrulefill}


\vspace{5mm}
\begin{center}
\begin{table}[htbp]
\centering
\caption{Rules of Inference for Quantified Statements}
\begin{tabular}{|l|l|}
\hline
Rule of Inference & Name \\
\hline
$\forall x P(x)$ & \\
$\therefore P(c)$ & Universal instantiation \\
\hline
$P(c)$ for an arbitrary $c$ & \\
$\therefore \forall x P(x)$ & Universal generalization \\
\hline
$\exists x P(x)$ & \\
$\therefore P(c)$ for some element $c$ & Existential instantiation \\
\hline
$P(c)$ for some element $c$ & \\
$\therefore \exists x P(x)$ & Existential generalization \\
\hline
\end{tabular}
\end{table}

\end{center}
\vspace{1.5in}


\begin{questions}
\question Identify the error or errors in this argument that supposedly shows that if $\exists x P(x) \land \exists x Q(x)$ is true, then $\exists x (P(x) \land Q(x))$ is true.
\begin{enumerate}
  \item $\exists xP(x) \lor \exists xQ(x)$ \textbf{Premise}
  \item $\exists xP(x)$ \textbf{Simplification from (1)}
  \item $P(c)$ \textbf{Existential instantiation from (2)}
  \item $\exists xQ(x)$ \textbf{Simplification from (1)}
  \item $Q(c)$ \textbf{Existential instantiation from (4)}
  \item $P(c) \land Q(c)$ \textbf{Conjunction from (3) and (5)}
  \item $\exists x(P(x) \land Q(x))$ \textbf{Existential generalization}
\end{enumerate}
\newpage
\newpage


\question Use rules of inference to show that if $\forall x(P(x) \lor Q(x))$ and $\forall x((\neg P(x) \land Q(x)) \rightarrow R(x))$ are true, then $\forall x(\neg R(x) \rightarrow P(x))$ is also true, where the domains of all quantifiers are the same.
\vspace{9in}


\question Use rules of inference to show that if $\forall x(P(x) \lor Q(x))$, $\forall x(\neg Q(x) \lor S(x))$, $\forall x(R(x) \rightarrow \neg S(x))$, and $\exists x\neg P(x)$ are true, then $\exists x\neg R(x)$ is true.

\vspace{9in}




\end{questions}
\end{document}