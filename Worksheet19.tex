\documentclass{exam}
\usepackage{amssymb}
\usepackage{amsmath}
\title{CS113/DISCRETE MATHEMATICS-SPRING 2024}
\author{Worksheet 19}
\date{Topic: Structural Induction}
\begin{document}
\maketitle

\begin{center}
\fbox{\fbox{\parbox{5.5in}{\centering Today, we will explore Structural Induction - a powerful proof technique which we use to establish the truth of statements about recursively defined structures or objects. It's an incredibly useful approach, especially when dealing with data structures like lists, trees, and graphs, or any other entities defined recursively. Happy Learning!}}}
\end{center}

\vspace{5mm}
\makebox[0.75\textwidth]{Student's Name and ID:\enspace\hrulefill}

\vspace{5mm}
\makebox[0.75\textwidth]{Instructor’s name:\enspace\hrulefill}

\vspace{5mm}
\begin{questions}


\question Show that the set S, defined by $1 \in S$ and $s + t \in S$ whenever $s \in S$ and $t \in S$, is the set of positive integers.

\vspace{9in}

\question Let $S$ be the set of positive integers defined by
Basis step: $1 \in S$.
Recursive step: If $n \in S$, then $3n + 2 \in S$ and $n^2 \in S$.

Show that if $n \in S$, then $n \equiv 1 \pmod{4}$.
\vspace{9in}


\question
Let S be the subset of the set of ordered pairs of integers defined recursively by:
Basis step: $(0, 0) \in S$.
Recursive step: If $(a, b) \in S$, then $(a + 2, b + 3) \in S$ and $(a + 3, b + 2) \in S$.

Use strong induction on the number of applications of the recursive step of the definition to show that 5 divides $(a + b)$ when $(a, b) \in S$.



\end{questions}
\end{document}

