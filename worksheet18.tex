\documentclass{exam}
\usepackage{amssymb}
\usepackage{amsmath}
\title{CS113/DISCRETE MATHEMATICS-SPRING 2024}
\author{Worksheet 18}
\date{Topic: Strong Induction}
\begin{document}
\maketitle

\begin{center}
\fbox{\fbox{\parbox{5.5in}{\centering Building on our previous knowledge of Strong Induction, we will now solve some more complex proofs. Happy Learning!}}}
\end{center}

\vspace{5mm}
\makebox[0.75\textwidth]{Student's Name and ID:\enspace\hrulefill}

\vspace{5mm}
\makebox[0.75\textwidth]{Instructor’s name:\enspace\hrulefill}


\vspace{5mm}
\begin{questions}

\question
Use strong induction to show that every positive integer
can be written as a sum of distinct powers of two, that is,
as a sum of a subset of the integers $2^0 =1, 2^1 =2, 2^2 =4$,
and so on. [Hint: For the inductive step, separately consider the case where k + 1 is even and where it is odd.
When it is even, note that $(k + 1)/2$ is an integer.]
\vspace{9in}

\question Suppose you begin with a pile of n stones and split this
pile into n piles of one stone each by successively splitting a pile of stones into two smaller piles. Each time you
split a pile you multiply the number of stones in each
of the two smaller piles you form, so that if these piles
have r and s stones in them, respectively, you compute
rs. Show that no matter how you split the piles, the sum
of the products computed at each step equals \( \frac{n(n-1)}{2} \).
\vspace{9in}

\question 
Prove by strong induction that Breaking a chocolate bar with $n \geq 1$ pieces into individual
pieces requires n – 1 breaks.


\vspace{9in}


\end{questions}
\end{document}
