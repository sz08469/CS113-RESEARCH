\documentclass{exam}
\usepackage{amssymb}
\usepackage{amsmath}
\title{CS113/DISCRETE MATHEMATICS-SPRING 2024}
\author{Worksheet 20}
\date{Topic: Structural Induction}
\begin{document}
\maketitle

\begin{center}
\fbox{\fbox{\parbox{5.5in}{\centering Let's continue our exploration of Structural Induction by engaging in more proof exercises to further solidify our understanding of this topic. Happy Learning!}}}
\end{center}

\vspace{5mm}
\makebox[0.75\textwidth]{Student's Name and ID:\enspace\hrulefill}

\vspace{5mm}
\makebox[0.75\textwidth]{Instructor’s name:\enspace\hrulefill}

\vspace{5mm}
\begin{questions}

\question
Consider the following recursively defined Set.\\
(i) $A \in S$ \\
(ii) If $x \in S$, then (x) in S.\\
Prove using Structural Induction that every element in S contains equal number of parentheses.
\vspace{9in}


\question 

Consider the following recursively defined Set.\\
(i) $6 \in S$, $15 \in S$ \\
(ii) If $x,y \in S$, then $x+y \in S$.\\
Prove using Structural Induction Show that every element of S is divisible by 3.
\vspace{9in}

\question Let j denote the empty string. Let A be any finite nonempty
set. A palindrome over A can be defined as a string that reads the same
forward as backward. For example, “mom” and “dad” are palindromes
over the set of English alphabets.\\
1. $j \in S$\\
2. $ \forall a \in A, a \in S$\\
3. $ \forall a \in A \forall x \in S, axa \in S$\\
4. All the elements in S must be generated by the rules above.\\
Prove by structural induction that S equals the set of all palindromes
over A.
\vspace{9in}


\end{questions}
\end{document}

