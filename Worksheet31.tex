\documentclass{exam}
\usepackage{tikz}
\usepackage{amssymb}
\usepackage{amsmath}
\usetikzlibrary{decorations.markings}

\title{CS113/DISCRETE MATHEMATICS-SPRING 2024}
\author{Worksheet 31}
\date{Topic: Cryptography }
\begin{document}
\maketitle
\vspace{5mm}
\begin{center}
\fbox{\fbox{\parbox{5.5in}{\centering Continuing our exploration of Crytography, we will learn RSA technique today which is a widely used public-key encryption technique in modern cryptography. Happy Learning!}}}
\end{center}
\vspace{5mm}

\makebox[0.75\textwidth]{Student's Name and ID:\enspace\hrulefill} 

\vspace{5mm}
\makebox[0.75\textwidth]{Instructor’s name:\enspace\hrulefill}

\vspace{5mm}




\begin{questions}

\question
Encrypt the message \texttt{ATTACK} using the RSA system with $n = 43 \cdot 59$ and $e = 13$, translating each letter into integers and grouping together pairs of integers.

\newpage



\question Encrypt the message \texttt{UPLOAD} using the RSA system with $n = 53 \cdot 61$ and $e = 17$, translating each letter into integers and grouping together pairs of integers.

\newpage


\question
What is the original message encrypted using the RSA system with $n = 53 \cdot 61$ and $e = 17$, if the encrypted message is 3185 2038 2460 2550? (To decrypt, first find the decryption exponent $d$, which is the inverse of $e = 17$ modulo $52 \cdot 60$.)


\newpage

\question
What is the original message encrypted using the RSA system with $n = 43 \cdot 59$ and $e = 13$, if the encrypted message is 0667 1947 0671? (To decrypt, first find the decryption exponent $d$, which is the inverse of $e = 13$ modulo $42 \cdot 58$.)
\newpage

\end{questions}
\end{document}

