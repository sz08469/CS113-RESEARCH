\documentclass{exam}
\title{CS113/DISCRETE MATHEMATICS-SPRING 2024}
\author{Worksheet 3}
\date{Topic: Predicates And Quantifiers}
\begin{document}
\maketitle

\begin{center}
\fbox{\fbox{\parbox{5.5in}{\centering Now, you will explore additional rules of equivalences involving quantifiers. Utilize these rules along with the previous rules to establish the validity of quantified statements.
Happy Learning!}}}
\end{center}

\vspace{5mm}
\makebox[0.75\textwidth]{Student's Name and ID:\enspace\hrulefill}

\vspace{5mm}
\makebox[0.75\textwidth]{Instructor’s name:\enspace\hrulefill}

\section{Table 1: Quantifiers. }
\vspace{5mm}
\begin{center}
\begin{tabular}{|c|c|c|}
  \hline
  Statement & When True? & When False? \\
  \hline
  $\forall x P(x)$ & $P(x)$ is true for every $x$. & There is an $x$ for which $P(x)$ is false. \\
  $\exists x P(x)$ & There is an $x$ for which $P(x)$ is true. & $P(x)$ is false for every $x$. \\
  \hline
\end{tabular}
\end{center}
\vspace{5mm}
\section{Table 2: De Morgan’s Laws for Quantifiers.}
\begin{center}
\begin{tabular}{|c|c|c|}
  \hline
  Negation & Equivalent Statement & When is Negation True? \\
  \hline
  $\lnot\exists x P(x)$ & $\forall x \lnot P(x)$ & For every $x$, $P(x)$ is false. \\
  $\lnot\forall x P(x)$ & $\exists x \lnot P(x)$ & There is an $x$ for which $P(x)$ is true. \\
  \hline
\end{tabular}
\end{center}
\vspace{5mm}


\begin{questions}
\question Suppose that the domain of the propositional function
P(x) consists of the integers 0, 1, 2, 3, and 4. Write out
each of these propositions using disjunctions, conjunctions, and negations.
\begin{parts}
\part
\(\exists x P(x)\)
\vspace{1in}

\part
\(\forall x P(x)\)
\vspace{1in}

\part
\(\exists x \lnot P(x)\)
\vspace{1in}

\part
\(\forall x \lnot P(x)\)
\vspace{1in}

\part
\(\lnot \exists x P(x)\)
\vspace{1in}

\part
\(\lnot \forall x P(x)\)
\vspace{1in}
\end{parts}


\question . Express the negation of these propositions using quantifiers, and then express the negation in English.
\begin{parts}
\part
Some drivers do not obey the speed limit.
\vspace{1in}

\part
All Swedish movies are serious.
\vspace{1in}

\part
No one can keep a secret
\vspace{1in}

\part
There is someone in this class who does not have a
good attitude.
\vspace{1in}
\end{parts}


\question Determine whether \(\forall x(P(x) \leftrightarrow Q(x))\) and \(\forall x P(x) \leftrightarrow \forall x Q(x)\) are logically equivalent. Justify your answer.
\vspace{9in}

\question Show that \(\forall x P(x) \lor \forall x Q(x)\) and \(\forall x (P(x) \lor Q(x))\) are not logically equivalent.

\vspace{9in}


\end{questions}
\end{document}