\documentclass{exam}
\usepackage{amssymb}
\title{CS113/DISCRETE MATHEMATICS-SPRING 2024}
\author{Worksheet 7}
\date{Topic: Types Of Proofs}
\begin{document}
\maketitle

\begin{center}
\fbox{\fbox{\parbox{5.5in}{\centering Today, we are excited to delve into the fascinating world of mathematical proofs! We will explore various types of proof techniques that mathematicians use to establish the truth of mathematical statements and propositions.Happy Learning!}}}
\end{center}

\vspace{5mm}
\makebox[0.75\textwidth]{Student's Name and ID:\enspace\hrulefill}

\vspace{5mm}
\makebox[0.75\textwidth]{Instructor’s name:\enspace\hrulefill}


\vspace{5mm}
\section{Different Types Of Proof And Their Approaches:}
\subsection{Direct Proof:}
A direct proof of a conditional statement p → q is constructed when the first step is the assumption that p is true; subsequent steps are constructed using rules of inference, with the final step showing that q must also be true. A direct proof shows that a conditional statement p → q is true by showing that if p is true, then q must also be true, so that the combination p true and q false never occurs. 
\subsection{Proof By Contraposition:}
An extremely useful type of indirect proof is known as proof by contraposition. Proofs
by contraposition make use of the fact that the conditional statement p → q is equivalent to its
contrapositive, ¬q → ¬p. This means that the conditional statement p → q can be proved by
showing that its contrapositive, ¬q → ¬p, is true
\subsection{Proof By Contradiction:}
To prove a statement by contradiction, we start by assuming the negation of the claim we want to prove. This assumption serves as a "proof by contradiction" assumption. We then proceed to show that this assumption leads to a contradiction or inconsistency. By doing so, we can conclude that the original claim must be true.
\subsection{Proof of Biconditionals:}
To prove a biconditional statement of the form  $p \iff q$, we need to establish the truth of both implications: P → Q and Q → P. This proof technique is known as a proof of biconditionals or a proof by double implication.

\vspace{1.5in}


\begin{questions}


\question Show that the square of an even number is an even number using a direct proof.
\vspace{4in}


\question Prove that if x, y, and z are integers and x + y + z is odd,
then at least one of x, y, and z is odd

\vspace{9in}
\question Show that if n is an integer and n3 + 5 is odd, then n is
even using:
\begin{parts}
\part
a proof by contraposition
\vspace{4in}
\part
a proof by contradiction
\vspace{8in}
\end{parts}

\question Prove the proposition P(0), where P(n) is the proposition
“If n is a positive integer greater than 1, then $n^2 > n$.”
What kind of proof did you use?
\vspace{9in}


\question Prove that if n is a positive integer, then n is odd if and
only if 5n + 6 is odd
\vspace{9in}



\end{questions}
\end{document}