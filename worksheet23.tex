\documentclass{exam}
\usepackage{tikz}
\usepackage{amssymb}
\usepackage{amsmath}
\title{CS113/DISCRETE MATHEMATICS-SPRING 2024}
\author{Worksheet 23}
\date{Topic: Matching, Complete Matching, and Marriage Hall's Theorem}
\begin{document}
\maketitle
\vspace{5mm}
\begin{center}
\fbox{\fbox{\parbox{5.5in}{\centering As we continue our exploration of bipartite graphs, we'll now dive into an exciting topic called "Matching." We'll learn about various types of matching, such as maximum matching and perfect matching, and also will learn Marriage Hall's Theorem. Happy Learning!}}}
\end{center}
\vspace{5mm}

\makebox[0.75\textwidth]{Student's Name and ID:\enspace\hrulefill} 

\vspace{5mm}
\makebox[0.75\textwidth]{Instructor’s name:\enspace\hrulefill}

\vspace{5mm}

\section{Some Important Terms:}
\subsection{Matching:}
In graph theory, a matching in an undirected graph is a subset of its edges in which no two edges share a common vertex. In other words, a matching is a set of edges such that each vertex is incident to at most one edge from the set. It can be seen as a collection of non-intersecting connections between vertices.

\subsection{Complete Matching:} 
A complete matching in an undirected graph is a special type of matching in which every vertex in the graph is incident to exactly one edge from the matching. In other words, a complete matching is a matching that covers all the vertices of the graph. In a bipartite graph, a complete matching connects every vertex from one set to its corresponding vertex in the other set, forming a one-to-one pairing.

\section{Hall's Marriage Theorem:}
The bipartite graph \( G = (V, E) \) with bipartition \((V1, V2)\) has a complete matching from \( V1 \) to \( V2 \) if and only if \( |N(A)| \geq |A| \) for all subsets \( A \) of \( V1 \).
\newpage

\begin{questions}

\question
Suppose that there are four employees in the computer
support group of the School of Engineering of a large university. Each employee will be assigned to support one
of four different areas: hardware, software, networking,
and wireless. Suppose that Ping is qualified to support
hardware, networking, and wireless; Quiggley is qualified to support software and networking; Ruiz is qualified
to support networking and wireless, and Sitea is qualified
to support hardware and software.
\begin{parts}
\part
Use a bipartite graph to model the four employees and
their qualifications.
\vspace{3in}

\part
Use Hall’s theorem to determine whether there is an
assignment of employees to support areas so that each
employee is assigned one area to support.
\newpage

\part
If an assignment of employees to support areas exists such that
each employee is assigned to one support area exists,
find one.
\vspace{3in}
\end{parts}
\question 
Suppose that there are five young women and six young
men on an island. Each woman is willing to marry some
of the men on the island and each man is willing to marry
any woman who is willing to marry him. Suppose that
Anna is willing to marry Jason, Larry, and Matt; Barbara
is willing to marry Kevin and Larry; Carol is willing to
marry Jason, Nick, and Oscar; Diane is willing to marry
Jason, Larry, Nick, and Oscar; and Elizabeth is willing to
marry Jason and Matt.
\begin{parts}
\part
Model the possible marriages on the island using a
bipartite graph.
\newpage

\part
Find a matching of the young women and the young
men on the island such that each young woman is
matched with a young man whom she is willing to
marry.
\vspace{3in}

\part
Is the matching you found in part (b) a complete
matching? Is it a maximum matching?
\newpage
\end{parts}


\question
Suppose that 2n tennis players compete in a round-robin
tournament. Every player has exactly one match with every other player during 2n - 1
consecutive days. Every
match has a winner and a loser. Show that it is possible
to select a winning player each day without selecting the
same player twice.(Hint: Use Marriage Hall's Theorem)
\newpage

\question
Suppose that m people are selected as prize winners in a
lottery, where each winner can select two prizes from a
collection of different prizes. Show if there are 2m prizes
that every winner wants, then every winner is able to
select two prizes that they want.
\newpage




\end{questions}
\end{document}

