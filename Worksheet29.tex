\documentclass{exam}
\usepackage{tikz}
\usepackage{amssymb}
\usepackage{amsmath}
\usetikzlibrary{decorations.markings}

\title{CS113/DISCRETE MATHEMATICS-SPRING 2024}
\author{Worksheet 29}
\date{Topic: Primes and Greatest Common Divisors }
\begin{document}
\maketitle
\vspace{5mm}
\begin{center}
\fbox{\fbox{\parbox{5.5in}{\centering In today's session, we will delve into two fundamental mathematical concepts: Primes and Greatest Common Divisors (GCD). Additionally, we will explore the Euclidean Algorithm, a historic and effective technique for determining the GCD by repeatedly subtracting the smaller number from the larger one until one of them reaches zero. Happy Learning!}}}
\end{center}
\vspace{5mm}

\makebox[0.75\textwidth]{Student's Name and ID:\enspace\hrulefill} 

\vspace{5mm}
\makebox[0.75\textwidth]{Instructor’s name:\enspace\hrulefill}

\vspace{5mm}




\begin{questions}

\question
Use the Euclidean algorithm to find:
\begin{parts}
    


\part gcd(100, 101)
\newpage


\part gcd(1529, 14038)
\newpage

\end{parts}


\newpage


\question Use the extended Euclidean algorithm to express
gcd(26, 91) as a linear combination of 26 and 91.

\newpage

\question
Use the extended Euclidean algorithm to express
gcd(144, 89) as a linear combination of 144 and 89.
\newpage

\question
Show that $a^m + 1$ is composite if $a$ and $m$ are integers
greater than 1 and $m$ is odd. 


\end{questions}
\end{document}

